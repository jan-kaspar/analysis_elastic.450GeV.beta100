\input macros

\def\baseDir{/afs/cern.ch/work/j/jkaspar/work/analyses/elastic/450GeV/beta100/4rp/}

%----------------------------------------------------------------------------------------------------

\hbox{}
\vskip-10mm

\centerline{\SetFontSizesXX Elastic analysis, $\sqrt s = 900\un{GeV}$, $\be^*_y = 100\un{m}$}
\vskip2mm
\centerline{\SetFontSizesXX analysis with 4 RPs }
\vskip2mm
\centerline{version: {\it \number\day. \number\month. \number\year}}

\vfil
\InsertToc

\vfil
\eject

\BeginText

%----------------------------------------------------------------------------------------------------
\chapter{Analysis approach}

\> RPs used (track required): 210-fr and 220-fr units in both arms



%----------------------------------------------------------------------------------------------------
\chapter[datasets]{Datasets}

\> LHC fills (Oct 2018)
\>> 7280: standard collimation, good quality
\>> 7281: crystal collimation, excellent quality
\>> 7282: standard collimation, good quality
\>> 7283: standard collimation, good quality
\>> 7284: standard collimation, good quality
\>> 7285: standard collimation, good quality
\>> 7286: standard collimation, bad quality
\>> 7287: standard collimation, bad quality
\>> 7288: standard collimation, bad quality
\>> 7289: crystal collimation, excellent quality
\>> 7290: standard collimation, bad quality
\>> 7291: crystal collimation, excellent quality

\> data quality can be judged from the following plots
\>> \plot{time_dependences/rate_vs_time.pdf}: good quality if rate in red (4-RP coincidence) is close to rate in blue (after all offline cuts)
\>> \plot{time_dependences/event_category_rates.pdf}: good quality if rate in red (track(s) + showers) stays close to rate in blue (single tracks)

\> trigger streams
\>> for main analysis: Totem1
\>> for efficiency studies: ZeroBias

\> input data selection
\>> as baseline: all good or excellent quality fills are used $\rightarrow$ fills 7286, 7287, 7288, 7290 are excluded
\>> fill 7282: two time periods (before beam cleaning) excluded due to significant $\th^*_y$ resolution degradation
\>>> 112100 to 112400
\>>> 115000 to 117000
\>> fill 7283: run 324467 excluded (too short)
\>> periods where horizontal RPs inserted excluded (degraded efficiency at larger t's)
\>>> fill 7281, run 324462: LS from 94 to 338
\>>> fill 7289, run 324532: LS from 399 to 520
\>>> fill 7291, run 324536: LS from 360 to 509



%----------------------------------------------------------------------------------------------------
\chapter[ntuples]{Ntuples}

\> current: /eos/totem/data/offline/2018/450GeV/beta100/Totem1/version3



%----------------------------------------------------------------------------------------------------
\chapter[beam-cond]{Beam conditions}

\> beam energy $E = 450\un{GeV}$, Lorenz $\ga \approx 479.7$

\> $\be^*_x = 70\un{m}$, $\be^*_y = 100\un{m}$

\> \plot{time_dependences/emittance.pdf}: normalised beam emittance as measured by BSRTs
\>> typical value: $\ep_N = 1.5\un{\mu m}$

\> single beam properties can be derived from normalised emittance $\ep_N$:
\eqref{
\si(\th^*_{x,\rm bd}) = \sqrt{\ep_N\over\ga \be^*_x}\ ,\qquad \si(x^*_{\rm sb}) = \sqrt{\ep_N \be^*_x\over \ga}
}{si be prop}
and likewise for $y$
\>> $\si(\th^*_{x,\rm bd})$ refers to angular fluctuations, i.e.~beam divergence
\>>> typical values: $6.7\un{\mu rad}$ (x), $5.6\un{\mu rad}$ (y)
\>> $\si(x^*_{\rm sb})$ corresponds to single-beam width
\>>> typical values: $470\un{\mu m}$ (x), $560\un{\mu m}$ (y)

\> the size of the beam spot (RMS of the 2-beam interaction vertex) can be related to the single-beam size:
\eqref{
\si(x^*) = \si(x^*_{\rm sb}) / \sqrt 2
}{vertex size}
and likewise for $y$
\>> typical values: $330\un{\mu m}$ (x), $395\un{\mu m}$ (y)

\> see Section \sref{resolution} for comparison of
\>> beam divergence estimated here to angular-reconstruction resolution
\>> vertex size estimated here to reconstructed vertex RMS

\> optical functions are give in Section \sref{optics}

\> bunches available
\>> 7280: 11, 520, 1019, 1541, 2067, 2555
\>> 7281: 11, 520, 1019, 1541, 2067, 2555
\>> 7282: 11, 520, 1019, 1541, 2067, 2555
\>> 7283: 11, 520, 1019, 1541, 2067, 2555
\>> 7284: 11, 1019, 1541, 2067, 2555
\>> 7285: 11, 520, 1019, 1541, 2067, 2555
\>> 7286: 11, 520, 1019, 1541, 2067, 2555
\>> 7287: 11, 520, 1019, 1541, 2067, 2555
\>> 7288: 11, 520, 1019, 1541, 2067, 2555
\>> 7289: 11, 520, 1019, 1541, 2067
\>> 7290: 11, 520, 1019, 1541, 2067, 2555
\>> 7291: 11, 520, 1019, 1541, 2067, 2555



%----------------------------------------------------------------------------------------------------
\chapter[alignment]{Alignment}


\section[alignment-det]{Determination}

\> strategy: standard procedure
\>> beam-based alignment: before data-taking
\>> track-based alignment: relative alignment between RP sensors
\>> alignment with elastics: absolute alignment wrt.~LHC beam

\> track-based alignment
\>> method: minimise track-hit residuals
\>> external analysis repository
\>> uncertainties
\>>> horizontal shift: $TODO\un{\mu m}$
\>>> vertical shift: $TODO\un{\mu m}$
\>>> rotation about $z$: $TODO\un{mrad}$

\> alignment with elastics
\>> method: illustrated in \plot{alignment/alignment_method.pdf}

\>> \plot{alignment/alignment_details.pdf}: detailed examples of the alignment extraction

\>> \plot{alignment/alignment_horizontal_overview.pdf}: details of horizontal-alignment fits for several fills and periods

\>> \plot{alignment/alignment.pdf}: result summary for all fills

\>> uncertainties
\>>> horizontal shift: $150\un{\mu m}$
\>>> vertical shift: $300\un{\mu m}$
\>>> rotation about $z$: $5\un{mrad}$

\> TODO: fine tuning with reco-check plots (also add to the strategy overview on top of this section)

\> induced uncertainties
\>> TODO
\iffalse
\>> shift in $\th_x^*$: $0.50\un{\mu rad}$ (single arm), $0.35\un{\mu rad}$ (double arm)
\>> shift in $\th_y^*$ (correlated top-bottom): $0.35\un{\mu rad}$ (single arm), $0.25\un{\mu rad}$ (double arm)
\>> shift in $\th_y^*$ (uncorrelated top-bottom): $0.017\un{\mu rad}$ (single arm), $0.012\un{\mu rad}$ (double arm)
\>> rotation about z: total uncertainty $2.24\un{mrad}$, two effects:
\>>> $\th_x^* \rightarrow \th_x^* + C \th_y^*$, $\si(C) = 0.013$ (single arm), $0.009$ (double arm)
\>>> $\th_y^* \rightarrow \th_y^* + D \th_x^*$, $\si(D) = 0.00039$ (single arm), $0.00028$ (double arm)

\section[alignment-val]{Validation}

\> \plot{alignment/angular_diff_vs_time.pdf}: left-right differences in reconstructed scattering angles
\>> typical value of $\De^{\rm R-L}\th^*_y$: $\approx 0.04\un{\mu rad}$

\> \plot{alignment/mean_th_x_vs_th_y.pdf}: mean of $\th_x^*$ as a function of $\th_y^*$ -- test for residual mis-rotations

\> \plot{alignment/mean_th_x_vs_time.pdf}:  mean of $\th_x^*$ as a function of time -- test for stability of $x$ alignment

\> observations (2 arm)
\>> shift in $\th_x^*$: $\De^{R-L} \th_x^*$ up to $1\un{\mu rad}$, the above estimate gives RMS of $0.5\un{\mu rad}$ thus at the edge of compatibility; maximum of $\De^{R-L} \th_x^*$ only occurs seldom; for safety take $0.5\un{\mu rad}$ for the two-arm uncertainty
\>> shift in $\th_y^*$: compatible with above estimate $0.29\un{\mu rad}$
\>> rotation about z: $\si(C) \approx 0.005$ significantly better than the above estimate, use this value
\>> 2D Gaussian fit of $\th^*_y$ vs $\th^*_x$ distributions combined from the two diagonals (program alignment\_final) gives centre positions
\>>> $|\th^*_x| < 0.03\un{\mu rad}$
\>>> $|\th^*_y| < 0.07\un{\mu rad}$

\fi

\> validation plots are discussed in Section \sref{reco-checks}



%----------------------------------------------------------------------------------------------------
\chapter[optics]{Optics}

\> currently, the nominal optics is used (confirmed to be OK by Frici on 26 Feb 2020)

\> optical functions (sent by Frici on 26 Feb 2020)

\centerline{\vbox{\halign{#\quad\hfil & $#$\quad\hfil & $#$\quad\hfil & $#$\quad & $#$\quad\hfil\cr\ln
unit & v_x & L_x\unp{m} & v_y & L_y\unp{m} \cr\ln
56-210-far & -2.29251110442821 & 35.1828162653735 & 0.126220649033512 & 171.911207102938\cr
56-220-far & -1.97644302902099 & 27.2787476702851 & 0.100459146707445 & 192.282747401847\cr
45-210-far & -2.24947329593445 & 35.8475945504415 & 0.124161170232800 & 173.630053228259\cr
45-220-far & -1.93196310652978 & 27.6759142870506 & 0.098769215019425 & 194.499649253117\cr
\ln
}}}

\> uncertainties -- not yet available
\iffalse
\>> 3 leading eigen modes reproduce the matrix sufficiently well
\>> propagated to relative uncertainties of scattering angles by program "reconstruction\_formulae/print\_optics\_uncertainties"
\>>> left-arm reconstruction: $\si(\de\th^*_x) = 0.00173$, $\si(\de\th^*_y) = 0.00152$
\>>> right-arm reconstruction: $\si(\de\th^*_x) = 0.00175$, $\si(\de\th^*_y) = 0.00152$
\>>> double-arm reconstruction: $\si(\de\th^*_x) = 0.00165$, $\si(\de\th^*_y) = 0.00152$, correlation factor $-90.29\un{\%}$

\>> single-arm perturbations decomposed to modes (program "systematics/optics\_scaling\_matrix")
\>>> relative errors, order: $\th^*_{x,L}$, $\th^*_{y,L}$, $\th^*_{x,R}$, $\th^*_{y,R}$
\>>> mode 1: -1.608E-03, +1.473E-03, -1.630E-03, +1.477E-03
\>>> mode 2: -5.157E-04, +2.541E-05, +5.566E-04, +2.746E-05
\>>> mode 3: +3.617E-04, +3.625E-04, +3.006E-04, +3.641E-04
\fi


\> validation plots are discussed in Section \sref{reco-checks}



%----------------------------------------------------------------------------------------------------
\chapter[hit-dist]{Hit distributions}

\> \plot{hit_distributions/hit_distributions.pdf}: hit distribution prior to elastic event selection

\> \plot{hit_distributions/hit_distributions_sel_cmp.pdf}: hit distribution with and without elastic event selection from Section \sref{cuts}



%----------------------------------------------------------------------------------------------------
\chapter[reco-form]{Reconstruction formulae}

\> method: MC simulation using parameters from Section \sref{beam-cond}

\> \plot{reconstruction_formulae/plot_formulae_graphs.pdf} : study of performance of various kinematics reconstruction formulae
\>> formulae
\>>> using different RPs as input
\>>> combining the input in various ways
\>> studying the impact of various effects: beam divergence, vertex, pitch, ...

\> the above study summarised in the following plots:
\>> \plot{reconstruction_formulae/plot_formulae_graphs_desc_th_x.pdf}: horizontal scattering angle
\>> \plot{reconstruction_formulae/plot_formulae_graphs_desc_th_y.pdf}: vertical scattering angle
\>> \plot{reconstruction_formulae/plot_formulae_graphs_desc_vtx_x.pdf}: horizontal vertex
\>> \plot{reconstruction_formulae/plot_formulae_graphs_desc_vtx_y.pdf}: vertical vertex

\iffalse
\> \plot{reconstruction_formulae/plot_formulae_correlation.pdf} : study of left-right correlations of angle reconstruction errors
\>> significant in $x$ due to neglecting $x^*$
\fi

\> angular formulae selected for analysis
\>> horizontal plane
\>>> single pot: $\displaystyle \th_x^{*} = \mp {x\over L_x}$
\>>> single arm: $\displaystyle {v_x^F x^N - v_x^N x^F\over v_x^F L_x^N - v_x^N L_x^F}$, resolution: $\approx 9\un{\mu rad}$ (beam divergence + pitch)
\>>> double arm: average of single arm, $\th_x^* = (\th_x^{*,R} + \th_x^{*,L}) / 2$
\>> vertical plane
\>>> single pot: $\displaystyle \th_y^{*} = \mp {y\over L_y}$
\>>> single arm: $\displaystyle \th_y^{*} = (\th_y^{*,F} + \th_y^{*,N}) / 2$, resolution: $\approx 5.6\un{\mu rad}$ (beam divergence)
\>>> double arm: average of single arm, $\th_y^* = (\th_y^{*,R} + \th_y^{*,L}) / 2$

\> vertex formulae selected for analysis
\>> horizontal plane
\>>> single arm: $\displaystyle {x^N L_x^F - x^F L_x^N\over v_x^N L_x^F - v^F L_x^N}$, resolution: $\approx 78\un{\mu m}$ (pitch)
\>>> double arm: L-R average, resolution: $\approx 55\un{\mu rad}$ (pitch)
\>> vertical plane
\>>> single arm: $\displaystyle {y^N L_y^F - y^F L_y^N\over v_y^N L_y^F - v^F L_y^N}$, resolution: $\approx 450\un{\mu m}$ (pitch)
\>>> double arm: L-R average, resolution: $\approx 320\un{\mu rad}$ (pitch)




%----------------------------------------------------------------------------------------------------
\chapter[cuts]{Cuts/elastic tagging}

\> event selection
\>> apply bunch selection
\>> apply LS selection
\>> require reco track in all 210-fr and 220-fr units of a diagonal
\>> elastic tagging, see below

\> elastic tagging: general cut structure $| a q_a + b q_b + c| < n_\si  \si$
\>> cut 1: $q_a = \th_x^{*R}$, $q_b = \th_x^{*L}$; \plot{cuts/cut_1.pdf}
\>> cut 2: $q_a = \th_y^{*R}$, $q_b = \th_y^{*L}$; \plot{cuts/cut_2.pdf}
\>> cut 5: $q_a = y^{R,2,F}$, $q_b = y^{R,2,F} - y^{R,1,F}$; \plot{cuts/cut_5.pdf}
\>> cut 6: $q_a = y^{L,2,F}$, $q_b = y^{L,2,F} - y^{L,1,F}$; \plot{cuts/cut_6.pdf}
\>> cut 7: $q_a = \th^*_{x}$, $q_b = x^{*,R} - x^{*,L}$; \plot{cuts/cut_7.pdf}
\>> cut 8: $q_a = \th^*_{y}$, $q_b = y^{*,R} - y^{*,L}$; \plot{cuts/cut_8.pdf}
\>> cut 9: $q_a = x^{R,2,F}$, $q_b = x^{R,2,F} - x^{R,1,F}$; \plot{cuts/cut_9.pdf}
\>> cut 10: $q_a = x^{L,2,F}$, $q_b = x^{L,2,F} - x^{L,1,F}$; \plot{cuts/cut_10.pdf}

\> all cuts applied at $n_\si = 4$ level

\> \plot{cuts/cuts_parameters_vs_time.pdf} : cut parameters as function of time

\> \plot{cuts/cuts_summary.pdf}: one-page example summarising all cuts for few selected fills



%----------------------------------------------------------------------------------------------------
\chapter[cut-eff-bckg]{Efficiency and purity of cuts, background}

\section{Background studies}

\> background: non-elastic events passing the tagging cuts

\> ``anti-diagonal`` data
\>> from configurations: 45 top -- 56 top, 45 bot -- 56 bot
\>> sign of $y$ swapped in sector 45 to process the data with the standard chain
\>> cannot contain any signal
\>> background expected similar as in diagonals

\> methods
\>> method 1: plot distributions of cut discriminators under various cut combinations, see e.g.\\ \plot{background/disc7_dist_cmp_cut_subset.pdf}
\>>> central part (signal): unaffected by cuts
\>>> tails (background): drops with increasing number of cuts
\>>> however: unknown interpolation of background from tails to the signal region
\>> method 2: plot distributions of cut discriminators also for anti-diagonal configurations, see\\ \plot{background/discs_dist_cmp_antidgn.pdf}
\>>> relatively good agreement in tails -- confirmation that the background is indeed similar in diagonals and anti-diagonals
\>>> anti-diagonals provide shape of interpolation to the signal region -- flat

\> results

\>> \plot{background/disc7_dist_cmp_cut_subset.pdf}: distributions of cut-7 discriminator, with different subsets of cuts
\>>> adding more cuts decreases the population in the tails while it keeps the central part (between the vertical dashed lines, signal) almost untouched

\>> as the above plot, but for several discriminators and per fill
\>>> \plot{background/discs_dist_cmp_cut_subset_DS-fill7280_Totem1.pdf}: fill 7280 (standard collimation, good quality)
\>>> \plot{background/discs_dist_cmp_cut_subset_DS-fill7281_Totem1.pdf}: fill 7281 (crystal collimation)
\>>> \plot{background/discs_dist_cmp_cut_subset_DS-fill7286_Totem1.pdf}: fill 7286 (standard collimation, bad quality)

\>> \plot{background/discs_dist_cmp_antidgn.pdf}: distributions of cut discriminators, after all available cuts, comparison between diagonal and anti-diagonal configurations
\>>> by comparing the signal (diagonals) and background (anti-diagonals) peak: rough estimate of $B/S = \O{TODO\un{\%}}$

\>> \plot{background,cut_efficiency/t_dist_antidgn_cmp.pdf}: $t$-distributions, after all available cuts, comparison between diagonal (mostly signal) and anti-diagonal (background) configurations
\>>> background / signal at $|t| \ls 0.2\un{GeV^2}$: per-mille level
\>>> background / signal at dip: up to $10\un{\%}$


\> additional material
\>> \plot{background/t_dist_cmp_cut_subset.pdf}: $t$-distributions under various cut combinations
\>> \plot{background/t_dist_cmp_antidgn.pdf}: comparison of $t$-distributions from diagonals and antidiagonals



\section{Study of signal loss due to cuts}

\> method
\>> plot $t$ distributions at different cut levels ($n_\si$)

\> \plot{background/t_dist_cmp_n_si.pdf}: $t$ distribution at different cut levels, for all fills
\>> qualitatively similar results from all fills

\> \plot{background/t_dist_cmp_n_si_binning.pdf}: $t$ distribution at different cut levels, for all binnings
\>> with fine binnings, bands visible
\>>> most likely because of the difference in bin contents is a small integer: 1, 2, 3, ...

\> results
\>> difference between $n_\si = 4$ (standard) and $5$ is below $TODO\un{\%}$ and almost flat -- TODO: verify



%----------------------------------------------------------------------------------------------------
\chapter[reco-checks]{Reconstruction checks}

\> method
\>> elastic events should have symmetries which can be checked in the data
\>> reconstruction from subsets of the RPs should give compatible results
\>> in particular, reconstructed scattering angles and vertex should be
\>>> compatible between arms
\>>> compatible between near and far RPs

\> interpretation of deviations
\>> deviations proportional to scattering angle are likely to be due to optics
\>> deviations independent of scattering angle are likely to be due to alignment

\> in case deviations are present: hopefully they are not time dependent

\> \plot{angular_distributions/ang_dist_diff_vs_angle_2D.pdf}: angular deltas as function of scattering angle, 2D histogram representation
\> \plot{angular_distributions/ang_dist_diff_vs_angle_prof.pdf}: angular deltas as function of scattering angle, profile representation

\> \plot{angular_distributions/ang_dist_mean_diff_vs_time.pdf}: angular deltas as function of time

\> \plot{angular_distributions/ang_dist_mean_th_x_vs_time.pdf}: mean value of horizontal scattering angle (maximum within RP acceptance) as function of time

\> \plot{angular_distributions/ang_dist_th_y_vs_th_x_cmp_fill.pdf}: vertical vs.~horizontal scattering angles, 2D histogram

\> \plot{angular_distributions/ang_dist_mean_th_x_vs_th_y.pdf}: mean horizontal scattering angle vs.~vertical scattering angle

\> TODO
\>> \plot{angular_distributions/ang_dist_diff_LR_vs_angle_vtx.pdf}
\>> \plot{angular_distributions/vtx_dist_vs_angle.pdf}



%----------------------------------------------------------------------------------------------------
\chapter[resolution]{Resolution}

\> resolution is typically studied by comparing kinematics of the elastic protons reconstructed in the left and right arm
\>> in the plots below, these results are typically drawn in red and blue (for the two diagonals)

\> in addition, the empirical resolution results can be compared to the prediction based on the measured emittance, see Section \sref{beam-cond} -- this prediction is drawn in black in the plots below

\> \plot{resolution/th_x_diffLR_vs_time.pdf}: time dependence of $\th^*_x$ resolution
\>>> black gives beam divergence component only
\>>> black significantly better than the observed resolution

\> \plot{resolution/th_y_diffLR_vs_time.pdf}: time dependence of $\th^*_y$ resolution
\>>> black gives beam divergence component only
\>>> black roughly as the observed resolution

\> \plot{resolution/vtx_x_diffLR_vs_time.pdf}: time dependence of $x^*$ resolution
\>> typical value $\si[\De^{R-L} x^*] \approx 500\un{\mu m}$
\>> according to Section \sref{reco-form}, it should be $\approx 110\un{\mu m}$, thus much better

\> \plot{resolution/vtx_y_diffLR_vs_time.pdf}: time dependence of $y^*$ resolution
\>> typical value $\si[\De^{R-L} y^*] \approx 2500\un{\mu m}$
\>> according to Section \sref{reco-form}, it should be $\approx 640\un{\mu m}$, thus much better

\> \plot{resolution/vtx_x_vs_time.pdf}: time dependence of reconstructed $x^*$ RMS (without subtracting the resolution)
\>> typical value $450\un{\mu m}$, after subtracting resolution ($250\un{\mu m}$) gives vertex size of $370\un{\mu m}$ which corresponds roughly to the black line and Section \sref{beam-cond}

\> \plot{resolution/vtx_y_vs_time.pdf}: time dependence of reconstructed $y^*$ RMS (without subtracting the resolution)
\>> typical value $1300\un{\mu m}$, after subtracting resolution ($1250\un{\mu m}$) gives vertex size of $360\un{\mu m}$ which corresponds roughly to the black line and Section \sref{beam-cond}

\> \plot{resolution/resolution_fits.pdf}: fits of the resolution time dependence as used in the acceptance correction, see Section \sref{acc-corr}



%----------------------------------------------------------------------------------------------------
\chapter[acc-corr]{Acceptance correction}

\section{Theory}

\vskip3mm

The smearing (beam, detector, ...) can be formally described as change of the true scattering angle, $\th_x$, to the reconstructed angle, $\th'_x$, (for brevity the stars are suppressed):
\eqref{
\th'_x = \th_x + \De \th_x
}{de th x}
The smearing may be different the left (L) and right (R) arm and can be decomposed as
\eqref{
\De\th_x^{\rm R} = m_x + {\d_x\over 2}\ ,\qquad \De\th_x^{\rm L} = m_x - {\d_x\over 2}\ ,
}{m d to th L R}
where $m_x$ modifies the L-R average while $d_x$ modifies the L-R difference:
\eqref{
m_x = {\De\th_x^{\rm R} + \De\th_x^{\rm L}\over 2}\ ,\qquad d_x = \De\th_x^{\rm R} - \De\th_x^{\rm L}\ .
}{th L R to m d}

The smearing and acceptance effects can then be described as transformation from true, $h_{\rm true}$, to reconstructed, $h_{\rm reco}$, probability distribution functions (PDFs):
\eqref{
\eqnarray{
	h_{\rm reco}(\th'_x, \th'_y) & = & \int \d\th_x \int \d\th_y\ h_{\rm true}(\th_x, \th_y) \cr
								 &  & \cdot \int \d m_x\ G_{m_x}(m_x) \int \d m_y\ G_{m_y}(m_y) \cr
								 &  & \cdot \int \d d_x\ G_{d_x}(d_x) \int \d d_y\ G_{d_y}(d_y) \cr
								 &  & \cdot \de\left( \th'_x - (\th_x + m_x) \right) \cr
								 &  & \cdot \de\left( \th'_y - (\th_y + m_y) \right) \cr
								 &  & \cdot \Th(\hbox{acc.~cond.}) \ ,\cr
}
}{sm acc model}
where the $G$ functions give distributions of the smearing parameters, thus the model assumes that the $m_{x,y}$ and $d_{x,y}$ parameters are independent (uncorrelated). The $\Th$ function expresses the acceptance condition. As the condition can be written in terms of $\th'_{x,y}$ and $d_{x,y}$, i.e.~the cuts are done in the smeared coordinates, \Eq{sm acc model} can be factorised as
\eqref{
	h_{\rm reco}(\th'_x, \th'_y) = h_{\rm smear}(\th'_x, \th'_y)\ F_{\rm acc}(\th'_x, \th'_y)\ .
}{sm acc model fac}
The first factor
\eqref{
\eqnarray{
	h_{\rm smear}(\th'_x, \th'_y) & = & \int \d\th_x \int \d\th_y\ h_{\rm true}(\th_x, \th_y) \cr
								 &  & \cdot \int \d m_x\ G_{m_x}(m_x) \int \d m_y\ G_{m_y}(m_y) \cr
								 &  & \cdot \de\left( \th'_x - (\th_x + m_x) \right) \cr
								 &  & \cdot \de\left( \th'_y - (\th_y + m_y) \right) \cr
}
}{sm model}
gives the PDF corresponding to the case where only smearing was applied. The effect of acceptance cuts is all contained in the second term:
\eqref{
\eqnarray{
	F_{\rm acc}(\th'_x, \th'_y) & = & \int \d d_x\ G_{d_x}(d_x) \int \d d_y\ G_{d_y}(d_y) \cr
								 &  & \cdot \Th(\hbox{acc.~cond.}) \ .\cr
}
}{acc model}

In the data reconstruction, one builds histogram(s) using weight $1 / F_{\rm acc}$ which according to \Eq{sm acc model fac} estimates $h_{\rm smear}$ wherever there is non-zero acceptance. A particularly interesting histogram is in $t$, or in equivalently in $\th$ which can be modelled as
\eqref{H_{\rm smear}(\th) = \int\limits_{\Om(\th)} \ \d\ph\ h_{\rm smear}(\th \cos\ph, \th\sin\ph)\ ,}{sm model th}
where $\Om(\th)$ is a set of $\ph$ values with non-zero acceptance. In order to recover the part with zero acceptance, one can introduce a correction
\eqref{H_{\rm smear,corr}(\th) = {1\over A_{\ph}(\th)} H_{\rm smear}(\th)\ ,}{sm model th corr}
\eqref{A_{\ph}(\th) = {1\over 2\pi} \int\limits_{\Om(\th)} \ \d\ph\ .}{A phi}
This correction assumes azimuthal ($\ph$) symmetry of $h_{\rm smear}$ which is typically slightly broken due to different smearing in $\th_x$ and $\th_y$. The error induced is typically absorbed into the ``unfolding correction'' which is applied in the final step of the analysis.

\section{Fiducial cuts}

\> there are two types of fiducial cuts
\>> separately in left and right arm
\>>> cuts corresponding to the actual limits (sensors, LHC)
\>>> they correspond to the condition $\Th$ in \Eq{acc model}
\>> cut in global (after averaging) $\th_{x, y}$
\>>> in order to avoid regions where $F_{\rm acc}$ is too small (and thus the corresponding weight too large)
\>>> they enter via the set $\Om$ in \Eq{sm model th}

\> all cuts have two bounds (illustrated in \plot{fiducial_selection/fid_sel_global.pdf})
\>> at low $|\th_y^*|$ (magenta line): mainly due to sensor edge
\>> at higher $|\th_y^*|$ (red line): mainly due to LHC apertures

\> all cuts are defined empirically by selecting a region within which one expects full acceptance ($\Th \equiv 1$ in \Eq{acc model}), shown in \plot{fiducial_selection/fid_sel_details.pdf}
\>> there is evident effect of the upstream 210-fr RPs which are tilted, causing inefficiency -- this region is cut out

\> \plot{fiducial_selection/fid_sel_cmp.pdf}: comparison of fiducial cuts


\section{Smearing correction}

\> ``smearing correction'' denotes the factor $1/F_{\rm acc}$
\>> it is calculated according to \Eq{acc model}, assuming Gaussian distributions of $d_{x,y}$, using numerical integration (GSL library)
\>> it is applied as described below \Eq{acc model}


\section{$\ph$ correction}

\> ``$\ph$ correction'' refers to $1 / A_{\ph}$
\>> it is calculated according to \Eq{A phi}
\>> it is applied according to \Eq{sm model th corr}


\> \plot{acceptance_correction/phi_symmetry.pdf} : verification of the assumed $\ph$ symmetry


\section{Program implementation, validation}

TODO

\iffalse

\> there are two implementations of \Eq{acc model}
\>> ``old''
\>>> assumes no dependence of acceptance condition of $d_x$, thus integrates over $d_y$ only
\>>> assumes Gaussian distribution of $d_y$
\>> ``new'': no assumptions needed, implemented as numerical integration using routines from GSL
\>>> for performance optimisation, this implementation {\it may} assume gaussianity of $d_y$

\> \plot{acceptance_correction/validation/validation_vs_old.pdf} : comparison of old and new implementations -- perfect agreement

\> \plot{acceptance_correction/validation/validation_gaussian_approx.pdf} : for the new implementation, comparison of results with and without the gaussian optimisation -- perfect agreement

\> below a study with MC is presented; two versions of simulation have been considered:
\>> ``corr'': including the correlation between $m_x$ and $d_x$ ($\rh \approx 0.2$) and correlation between $m_y$ and $d_y$ ($\rh\approx 0.02$)
\>> ``uncorr'': disregarding the correlations
\>> it will be shown that the correlation has no significant effect

\> \plot{acceptance_correction/validation/validation_with_mc_th_y_acc.pdf} : smearing acceptance as a function of $\theta^*_y$ for 3 bands of $\th^*_x$
\>> MC (violet) vs.~calculated correction (black): perfect agreement

\> \plot{acceptance_correction/validation/validation_with_mc_th_y_hist_after_corr.pdf} : for 3 bands of $\th^*_x$, ratio of $\theta^*_y$ distributions -- with smearing only and with smearing+fiducial cuts+smearing acceptance correction
\>> perfectly compatible with 1

\> \plot{acceptance_correction/validation/validation_with_mc_t_dist_cmp.pdf} : comparison of $t$-distributions
\>> black and red histograms perfectly compatible

\> \plot{acceptance_correction/validation/validation_with_mc_t_unsmear_corr.pdf} : comparison of "unsmearing" correction from this MC study (red, blue) and from a dedicated calculation (black)
\>> perfect compatibility

\fi

\section{Results}

\> \plot{acceptance_correction/acc_corr_hists.pdf}: components of the acceptance correction plotted as a function of $t_y$ or $t$

\> \plot{acceptance_correction/acc_corr_t_dist.pdf}: effect of the acceptance correction on the $t$-distribution



%----------------------------------------------------------------------------------------------------
\chapter[efficiency]{Efficiency studies}

\iffalse

\vskip3mm
Each event is weighted with the following correction factor
\eqref{
	{1\over 1 - I_{\rm trig}}\ 
	{1\over 1 - I_{\rm DAQ}}
	{1\over 1 - I_{\rm PU}}\ 
	{1\over 1 - I_{1RP} - I_{\rm 2RP}}\ ,
}{eff corr}
where symbols $I$ stand for various inefficiencies discussed below.

\section[efficiency-trigger]{Trigger efficiency}

\> method:
\>> take zero-bias (BX) data
\>>> using bit 9 (e.g.~https://twiki.cern.ch/twiki/bin/view/TOTEM/90m)
\>>> i.e.~ev.trigger\_bits \& 512
\>> select elastic events (standard tagging) $\rightarrow$ number of events $N({\rm BX,elastic})$
\>> out of the selected events, check how many have trigger flag on $\rightarrow$ number of events $N({\rm BX,elastic,trigger})$
\>>> trigger flag = ev.trigger\_bits \& 7 (i.e.~any of the bits 0, 1 and 2)
\>> then trigger efficiency is obtained as follows:

$$1 - I_{\rm trig} = {N({\rm BX,elastic,trigger}) \over N({\rm BX,elastic})}$$

\> results:
\>> fill 5317, diagonal 45 bot -- 56 top: $N({\rm BX,elastic}) = N({\rm BX,elastic,trigger}) = 7900$
\>> fill 5317, diagonal 45 top -- 56 bot: $N({\rm BX,elastic}) = N({\rm BX,elastic,trigger}) = 7177$

\> this correction is not applied since it does not modify $t$-distribution shape


\section[efficiency-trigger]{DAQ efficiency}

\> method: per time slice calculate from the trigger block of raw data:
$$\hbox{DAQ efficiency} = {N(\hbox{recorded events})\over N(\hbox{triggered events})} \equiv 1 - I_{\rm DAQ}$$

\> \plot{efficiencies/daq_efficiency.pdf}: DAQ efficiency as function of time

\> this correction is not applied since it does not modify $t$-distribution shape



\section[efficiency-pileup]{Pile-up}

\> method
\>> take BX sample, split it into time slices
\>> per RP, evaluate frequency of ``destructive`` signal that would break reconstruction if piled up with an elastic event

$$I^{i}_{\rm PU} = {N(\hbox{destructive signal in RP $i$})\over N(\hbox{anything})}$$

\>> typical conditions of ``destructivity``
\>>> ``pl\_suff'': sufficient number of planes is on
\>>> ``pat\_suff'': at least one U or V pattern is recognised 
\>> use OR between left and right arm -- it is sufficient to loose any arm to loose the elastic event

$$I_{\rm PU} = {N(\hbox{destructive signal in any 220-fr RP})\over N(\hbox{anything})}$$


\> \plot{efficiencies/pileup_details.pdf}: contributions from each arm and their combination, per diagonal

\> \plot{efficiencies/pileup.pdf}: final inefficiency as a function of time

\> this correction is not applied since it does not modify $t$-distribution shape


\section[efficiency-uncorrelated]{Uncorrelated single RP inefficiencies (3-out-of-4)}

\> method: derive from results of the 4-RP analysis
\>> transfer validation: \plot{efficiencies/eff3outof4_fits_summary_cmp_4rp_anal.pdf}
\>> take the results for 220-fr pots only

$$I_{\rm 1RP}(\th_y^*) = \sum\limits_{i \in\hbox{\SmallerFonts 220-fr}} I_{\rm 1RP}^i$$

\> typical form of $I_{\rm 1RP}(\th_y^*)$: linear in $\th^*_y$
\>> value at $\th_y^* = 0$: typically about $6.9\un{\%}$
\>> slope: typically about $150\un{rad^{-1}}$

\> uncertainties of $I_{\rm 1RP}(\th_y^*)$
\>> value at $\th_y^* = 0$: $0.3\un{\%}$
\>> slope: $15\un{rad^{-1}}$

\> this correction is applied since it modifies the $t$-distribution shape


\section[efficiency-correlated]{Correlated RP inefficiencies}

\> typical case: shower initiated upstream (210-far) propagates downstream (220-far)
\>> this is not included in the uncorrelated single-RP inefficiency above

\> method: external study (Geant4, data driven check at $\sqrt s = 8\un{TeV}$)
\>> inefficiency per arm: $(1.5 \pm 0.7)\un{\%}$ (e.g.~from 1km or non-exponential paper)
\>> total inefficiency (two arms):

$$I_{2RP} = (3 \pm 1)\un{\%}$$

\> this correction is applied since it modifies the $t$-distribution shape

\fi



%----------------------------------------------------------------------------------------------------
\chapter[unfolding]{Unfolding of resolution effects}

\> method
\>> take a fit of data with model including both Coulomb and hadronic contributions
\>> this fits serves as an input (truth) to simulation with and without smearing effects
\>> ratio of these two curves gives unfolding correction
\eqref{U(\th) = {H_{\rm smear,corr}(\th)\over h_{\rm true}(\th)}}{unfold corr}
where the numerator comes from \Eq{sm model th corr} and therefore this correction takes into account the small bias introduced in acceptance correction handling
\>>> in practice the correction $U$ is evaluated as a function of $t$

\> there are two evaluations of \Eq{unfold corr}
\>> numerical integration: fast, accurate
\>> Monte Carlo: very simple implementation, needs long computing time for reasonable accuracy
\>> \plot{unfolding/unfolding_cmp_mc_ni.pdf} : comparison of the two methods -- very good agreement

\> \plot{unfolding/unfolding_ni_model_cmp.pdf} : dependence on the initial data fit
\>> TODO: check when more models available

\> results -- as expected
\>> effects in general small -- very small beam divergence
\>> effects only present at low $|t|$ (strong non-linearity due to Coulomb) and at dip region

\> uncertainties
\>> values of $\si(m_x)$ and $\si(m_y)$ taken from Section~\sref{resolution}, end of bullet on $m_{x,y}$
\>>> TODO
\>> model dependence: TODO: comment, see \plot{unfolding/unfolding_ni_model_cmp.pdf}


%----------------------------------------------------------------------------------------------------
\chapter[normalisation]{Normalisation}

The datasets are normalised such that
$$\int_{t_{\rm min}}^{t_{\rm max}} {\d\si^{\rm dataset}\over \d t} = \int_{t_{\rm min}}^{t_{\rm max}} {\d\si^{\rm Coulomb}\over \d t}$$
where $|t|_{\rm min} \approx 2\cdot10^{-4}\un{GeV^2}$ and $|t|_{\rm min} = 8\cdot10^{-4}\un{GeV^2}$. In this $t$-range, the about $10\un{\%}$ of the complete cross-section is expected not to come from the Coulomb interaction.



%----------------------------------------------------------------------------------------------------
\chapter[binning]{Binning}

\iffalse

\> method: binning ``ob-$\langle n\rangle$-$\langle u\rangle$-$\langle w_m\rangle$`` is built as follows
\>> at low $|t|$ ($|t| \ls 0.4\un{GeV^2}$): bin size about $n\times \si(\hbox{$t$ smearing})$
\>> at mid $|t|$: bin size for a fixed statistical uncertainty $u\un{\%}$
\>> if needed at large $|t|$: constant bin size $w_m$ (in $\rm GeV^2$) to avoid excessively large bins

\> binnings used in analysis
\>> ob-1-20-0.05
\>> ob-2-10-0.05
\>> ob-3-5-0.05
\fi

\> \plot{binning/bin_size_vs_t.pdf} : visualisation of binning determinants vs.~several binnings used in the analysis

\> temporarily a very fine binning ``eb'' is used -- to spot potential local problems



%----------------------------------------------------------------------------------------------------
\chapter[validation]{Validation of analysis chain}

\iffalse

\> method
\>> simulate data in the same format (distilled ntuple) as the LHC data
\>> run exactly the same analysis chain as for the LHC data
\>> compare MC truth with analysis results

\> MC simulation (program ``simulate\_distill``)
\>> inverse-CDF generation of $t$ according to a data fit including both hadronic and Coulomb components
\>> simulated: vertex smearing, beam divergence, proton transport (linear), RP resolution
\>> time-dependence of vertex RMS: linear from $530$ to $630\un{\mu m}$
\>> simulated 3-out-of-4 and 2-out-of-4 inefficiency, $\th^*_y$ dependent
\>>> according to the fit copied from the 4-RP analysis (the same as used in the analysis)

\> analysis
\>> the same configuration as for the LHC data (``parameters\_global.h``, the same cuts)
\>> time-dependent resolution curves determined from the simulated sample (by program ``resolution\_fit``)
\>> normalisation determined as for LHC data (by program ``normalisation/normalisation``)

\> results
\>> \plot{chain_validation/resolutions_vs_time.pdf} : resolution as function of time
\>>> as expected: linear dependence, very small effect for $\th_y^*$ (almost parallel-to-point focusing)
\>> \plot{chain_validation/t_dist_cmp_truth.pdf} : comparison of $t$-distributions, MC truth vs.\ reconstructed
\>>> perfect agreement

\fi



%----------------------------------------------------------------------------------------------------
\chapter[t-distributions]{$t$-distributions}

\> \plot{t_distributions/t_dist_fill_cmp.pdf} : comparison of $t$-distributions from different fills and diagonals

\> \plot{t_distributions/t_dist_merged_cmp_dgn.pdf} : merged $t$-distribution, per diagonal

\> \plot{t_distributions/t_dist_merged_cmp_binning.pdf} : merged and combined $t$-distribution in several binnings

\iffalse
\> \plot{t_distributions/t_dist_cmp_bunch.pdf} : comparison of $t$-distributions from different bunches (fill 5317)

\> \plot{t_distributions/t_dist_rel.pdf} : merged $t$-distribution in a relative reference frame
\fi



%----------------------------------------------------------------------------------------------------
\chapter[systematics]{Systematic uncertainties}

\iffalse

\section[systematics-general]{General notes}

\> method
\>> per diagonal
\>>> assume a true $t$-distribution (fit of final data)
\>>> do simulation and reconstruction, with and without bias
\>>> apply unsmearing correction and normalisation procedure
\>>> compare results with and without bias $\Rightarrow$ systematic effect given by the ratio of the 2 distributions
\>>> respect the correlations in procedure: e.g.~in case of misalignment, also the fiducial cuts move
\>> diagonal combination
\>>> effectively the same merging as for data
\>>> respects the correlations between diagonals

\> 2 methods of simulation
\>> Monte-Carlo (MC): simple implementation, slow ($\gs 10^9$ samples needed), works for all bias types
\>> numerical integration (NI)
\>>> for many bias types fast -- the 4D integral ($m_x, m_y, d_x$ and $d_y$) can be factorised into two 2D integrals ($m$'s and $d$'s)
\>>> more complicated implementation
\>>> not implemented for the bias types where the factorisation does not work

\> two steps of effect propagation
\>> analytic propagation to the building blocks: $\th^{*L}_x$, $\th^{*R}_x$, $\th^{*L}_y$, $\th^{*R}_y$ and event weight (corrections, normalisation, ...)
\>> numerical propagation: from building blocks to $t$-distribution (either MC or NI)

\> two types of effects on $\th^{*L,R}_{x, y}$
\>> left-right symmetric: vary the value of $t$, do not change the value of acceptance correction
\>> left-right anti-symmetric: do not vary the value of $t$, change the value of acceptance correction


\section[systematics-effects]{Effects included}

\> alignment: shifts in $\th^*_{x,y}$
\>> considered both LR symmetric and anti-symmetric modes
\>> in the vertical plane: both correlated and uncorrelated between diagonals
\>> uncertainty values presented in Section~\sref{alignment-det}
\>>> for vertical LR anti-symmetric mode, the uncertainty is extracted from data, see Section~\sref{alignment-val}

\> $x-y$ tilts: mixing between $\th^*_{x}$ and $\th^*_{y}$
\>> considered both LR symmetric and anti-symmetric
\>> uncertainty values presented in Section~\sref{alignment-det}

\> optics: scaling of $\th^*_{x,y}$ based on covariance matrix by Frici
\>> includes correlations between $x$ and $y$, L and R
\>> 3 modes (out of 4) found important
\>> uncertainty values presented in Section~\sref{optics-matched}

\> (smearing) acceptance correction
\>> uncertainty of $\si(d_x)$, $\si(d_y)$; see Section~\sref{resolution}, bullet on $d_{x,y}$
\>> non-gaussianity of $d_x$ distribution (in $d_y$ no relevant non-gaussianity observed); see Section~\sref{resolution}, end of bullet ``non-gaussianity''
\>> beam divergence L-R asymmetry, see Section~\sref{resolution}, bullet on correlation between $d$ and $m$

\> inefficiency correction
\>> only relevant from uncorrelated 3/4, Section~\sref{efficiency-uncorrelated}, and correlated upstream-downstream, Section~\sref{efficiency-correlated}
\>> both slope and intercept considered
\>> uncertainty values presented in the quoted sections

\> beam momentum
\>> considered relative uncertainty $\De p/p = 0.1\un{\%}$

\> unsmearing
\>> uncertainty of $\si(m_x)$, $\si(m_y)$
\>> fit-model dependence
\>> uncertainty values presented in Section~\sref{unfolding}

\> normalisation
\>> consider uncertainty of $5\un{\%}$


\section[systematics-results]{Results}

\> \plot{systematics/cmp_mc_ni.pdf} : individual effects, per diagonal, two views (low-$|t|$ and full range)
\>> very good agreement between MC and NI calculations

\> \plot{systematics/dgn_combination.pdf} : individual effects, combination from the two diagonals
\>> for effect with no correlation between diagonals, there are two combined modes, each corresponding to one diagonal

\> \plot{systematics/dgn_combination_summary.pdf} : combination from the two diagonals, all contributions in one plot, several $|t|$ ranges

\> \plot{systematics/dgn_combination_matrix.pdf} : correlation matrix for relative uncertainties

\> leading contributions at very low $|t|$
\>> vertical alignment
\>> uncertainty of vertical beam divergence

\> leading contributions at medium $|t|$ ($< 0.2\un{GeV^2}$)
\>> beam momentum
\>> optics mode 3

\> full uncertainty at low $|t|$ -- comparison to $1\un{km}$
\>> $1\un{km}$: $\approx 6\un{\%}$, here $\approx 1.7\un{\%}$
\>> in both cases the leading uncertainty is vertical misalignment
\>> for $1\un{km}$ only 1 diagonal available at low $|t|$ and thus the compensation between diagonal can not occur

\fi


%----------------------------------------------------------------------------------------------------

\EndText
\bye
