\input book

%----------------------------------------------------------------------------------------------------

\def\_{\leavevmode \kern.06em \vbox{\hrule width.5em}}

\def\TODO#1{\cRed{TODO: #1}\cBlack}

%----------------------------------------------------------------------------------------------------

\def\parseGet#1_#2;{%
	\def\parseCurr{#1}%
	\def\parseBuf{#2}%
}

\def\parseLoop{%
	\expandafter\parseGet\parseBuf;%
	\ifx\parseOutput\empty
		\edef\parseOutput{\parseCurr}%
	\else
		\edef\parseOutput{\parseOutput\_\parseCurr}%
	\fi
	\ifx\parseBuf\empty
	\else
		\parseLoop
	\fi
}

\def\parse#1{%
	\def\parseBuf{#1_}%
	\def\parseOutput{}%
	\parseLoop
	\parseOutput
}

%----------------------------------------------------------------------------------------------------

\gdef\plot#1{%
	\leavevmode
	\pdfstartlink
		attr{/Border [0 0 0]}%
		user {%
	    /Subtype /Link
	    /A << 
	        /Type /Action 
	        /S /URI 
	        /URI (\baseDir/plots/#1) 
	    >>}%
	\cBlue\uline{\parse{#1}}\cBlack
	\pdfendlink
}

%----------------------------------------------------------------------------------------------------

\gdef\summary#1{%
	\leavevmode
	\pdfstartlink
		attr{/Border [0 0 0]}%
		user {%
	    /Subtype /Link
	    /A << 
	        /Type /Action 
	        /S /URI 
	        /URI (\baseDir/summaries/#1) 
	    >>}%
	\cBlue\uline{\parse{#1}}\cBlack
	\pdfendlink
}

%----------------------------------------------------------------------------------------------------
%----------------------------------------------------------------------------------------------------

\def\InsertToc{%
	\def\currentChapterName{Contents} \def\currentPartName{}%
	\pdfdest name {TOC} xyz
	\pdfoutline goto name {TOC} count 0 {\currentChapterName}%
	\input\jobname.toc
}

%----------------------------------------------------------------------------------------------------

\def\chapterBase#1#2{%
	\advance\nchapter1
	\nsection=0
	\nsubsection=0
	\nssubsection=0
	\eqn=0
	\tabn=0
	\fign=0
	%
	%\vfil\eject\forceoddpage%
	\ifnum\nchapter<65
		\edef\currentChapterNumber{\the\nchapter}%
		\edef\currentChapterName{Chapter \currentChapterNumber\ #2}%
	\else
		\expandafter\GetNextLetter\alphabet;
		\edef\currentChapterNumber{\letter}%
		\edef\currentChapterName{Appendix \currentChapterNumber\ #2}%
	\fi
	\edef\currentPartNumber{\currentChapterNumber}%
	\edef\currentPartName{\currentChapterNumber\ #2}%
	%
	\TOCwrite{\TOCline}{0}{\currentChapterNumber}{#2}{\the\pageno}%
	%
	\pdfdest name {sn:\currentChapterNumber} xyz
	\pdfoutline goto name {sn:\currentChapterNumber} count \GetOutlineCount{\currentChapterNumber} {\currentPartName}%
	%
%%	\vbox to0pt{}
%%	\hbox{%
%%		\vtop{%
%%			\noindent\leftskip0pt plus 1fil\parfillskip0pt
%%			\fChapterWord
%%			\ifnum\nchapter<65
%%				Chapter \currentChapterNumber
%%			\else
%%				Appendix \currentChapterNumber
%%			\fi
%%			\vskip3mm
%%			%\hbox{\advance\hsize-10cm\\hbox to10cm{\hrulefill}}%
%%			\line{\hskip9cm\hrulefill}%
%%			\vskip3mm
%%			\fchapter\baselineskip30pt #2
%%		}%
%%		\ifShowLabels
%%			\rlap{\labelColor\quad#1\cBlack}%
%%		\fi
%%	}%
%%	\vskip2cm
%%	\mark{}%
	\parindent = 0pt
	\everypar={\parindent=\ParIndent \everypar={}}%
		\penalty-\clubpenalty
		\vskip1\baselineskip
		\bgroup
			\baselineskip\blschapter
			\hbox{%
				\vbox{%
					\rightskip0pt plus1fill
					\titskip
					\bgroup
						\fchapter
						\currentPartName
					\egroup
				}%
				\ifShowLabels
					\rlap{\labelColor#1\cBlack}%
				\fi
			}%
		\egroup
		\mark{}%
		\penalty\clubpenalty
		%\vskip0.5\baselineskip

	%
	\NewChaptertrue
}

%----------------------------------------------------------------------------------------------------

\pdfcatalog{/PageMode /UseOutlines}

\let\BiggerFonts\SetFontSizesXII
\let\NormalFonts\SetFontSizesX
\let\SmallerFonts\SetFontSizesVIII

\font\fchapter 		= pplb8z at 15pt
\font\fsection	 	= pplb8z at 12pt
\font\fsubsection 	= pplb8z at 10pt
\font\fssubsection	= pplb8z at 10pt

\ea\def\csname toc width 0\endcsname{6mm}
\ea\def\csname toc width 1\endcsname{10mm}
\ea\def\csname toc width 2\endcsname{10mm}
\ea\def\csname toc width 3\endcsname{14mm}

\ea\def\csname toc indent 0\endcsname{0mm}
\ea\def\csname toc indent 1\endcsname{6mm}
\ea\def\csname toc indent 2\endcsname{16mm}
\ea\def\csname toc indent 3\endcsname{26mm}

\NormalFonts

\Reftrue
\Toctrue
